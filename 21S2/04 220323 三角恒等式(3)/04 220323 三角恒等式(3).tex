%!TEX TX-program = xelatex
\documentclass[8pt]{article}
\usepackage{allan-eason}

\usetikzlibrary{positioning}
\usetikzlibrary{svg.path}

\graphicspath{ {./images/} }


\author{\normalfont\sffamily\large\bfseries{高一(4)班\ 邵亦成\ 48号}}
\title{\normalfont\sffamily\huge\bfseries{\textcolor{allanblue}{220323}\ \textcolor{allancyan}{三角恒等式(3)}\ 参考答案与题目选解}}
\date{}

\geometry{a4paper, scale=0.8}

\lhead{220323\ 三角恒等式(3)\ 参考答案与题目选解}

\begin{document}

	\maketitle

	\section{填空题}
		\defword{1. }在$\left( -1080\degree, -360\degree \right)$中与$-35\degree$终边相同的角的集合是 \answer{$\left\{-395\degree, -755\degree\right\}$}.

		\defword{解析. }角的终边.
		~\\

		\defword{2. }已知$\sin \left(\alpha + \dfrac{\pi}{6}\right)=\dfrac{5}{13}$, 则$\cos\left(\dfrac{4\pi}{3}-\alpha\right)=$\answer{$-\dfrac{5}{13}$}.

		\defword{解析. }诱导公式.
		~\\

		\defword{3. }已知$f(\cos x)=\cos 3x$, 则$f\left(\sin 30\degree\right)=$ \answer{$-1$}.

		\defword{解析. }诱导公式.
		~\\

		\defword{4. }若$\dfrac{\cos 2\alpha}{\sin \left(\alpha - \dfrac{\pi}{4}\right)}=-\dfrac{\sqrt{2}}{2}$, 则$\cos \alpha + \sin \alpha=$ \answer{$\dfrac{1}{2}$}.

		\defword{解析. }两倍角公式, 诱导公式.

		$$\cos 2\alpha = \cos^2 \alpha - \sin^2 \alpha, \sin\left(\alpha - \dfrac{\pi}{4}\right)=\dfrac{\sqrt{2}}{2}(\sin\alpha - \cos\alpha)\Rightarrow \YS = -\sqrt{2}(\sin \alpha + \cos \alpha) \Rightarrow \sin \alpha + \cos \alpha = \dfrac{1}{2}.$$
		~\\

		\defword{5. } 已知$\dfrac{\tan \theta}{\tan \theta - 1}=-1$, 则$\dfrac{\sin(\pi - \theta)-3\sin\left(\dfrac{\pi}{2} + \theta\right)}{\cos \left(\theta - \dfrac{5\pi}{2}\right)-\cos(-3\pi + \theta)}=$ \answer{$-\dfrac{5}{3}$}.

		\defword{解析. }诱导公式.
		~\\

		\defword{6. } 把下式化为$A\sin(\omega x + \varphi) (A>0, \omega>0)$的形式: $-\dfrac{5}{3}\sin 2x -\dfrac{5}{3}\cos 2x=$ \answer{$\dfrac{5}{3}\sqrt{2} \sin\left(2x + \dfrac{5}{4}\pi\right)$}; $6\cos 3x - 2\sqrt{3} \sin3x =$ \answer{$4\sqrt{3} \sin\left(3x+\dfrac{2}{3}\pi\right)$}.

		\defword{解析. }辅助角公式, 诱导公式.
		~\\

		\defword{7. } 若$\dfrac{\abs{\sin \alpha}}{\sin\alpha}+\dfrac{\cos\alpha}{\abs{\cos\alpha}}=0$, 试判断$\cot\left(\sin\alpha\right)\cdot\tan\left(\cos\alpha\right)$的符号. \answer{负}.

		\defword{解析. }象限角三角比正负性.
		~\\

		\defword{8. } 已知$\alpha, \beta \in \left(\dfrac{3\pi}{4}, \pi\right)$, $\sin(\alpha+\beta)=-\dfrac{3}{5}, \sin\left(\beta - \dfrac{\pi}{4}\right)=\dfrac{12}{13}$, 则$\cos\left(\alpha + \dfrac{\pi}{4}\right)=$ \answer{$-\dfrac{56}{65}$}.

		\defword{解析. }差角公式.
		~\\

		\defword{9. } 已知$\tan \dfrac{\alpha}{2}=-2$, 则$\sin^2\left(\dfrac{\alpha}{2}-\dfrac{\pi}{4}\right)=$ \answer{$\dfrac{9}{10}$}.

		\defword{解析. }同角三角比.
		~\\

		\defword{10. }写出正确的序号. \answer{(1)}.
			\begin{enumerate}[label=(\arabic*)]
				\item $\sin 5\theta + \sin 3\theta = 2\sin 4\theta \cos\theta$.
				\item $\cos 3\theta - \cos 5\theta = -2\sin 4\theta \sin\theta$. \answer{$2\sin 4\theta \sin \theta$}.
				\item $\sin 3\theta - \sin 5\theta = -\dfrac{1}{2}\cos 4\theta \cos \theta$. \answer{$-2\cos 4\theta \sin \theta$}.
				\item $\sin 5\theta + \cos 3\theta = 2\sin 4\theta \cos \theta$. \answer{$2\sin\left(\theta + \dfrac{\pi}{4}\right)\cos\left(4\theta-\dfrac{\pi}{4}\right)$}.
			\end{enumerate}

		\defword{解析. }和差化积公式, 诱导公式.

	\section{解答题}
		\defword{11. }化简:
			$$\frac{1+\sin x}{\cos x} \cdot \left[\frac{\sin 2x}{2\cos^2\left(\frac{\pi}{4}-\frac{x}{2}\right)}-\frac{\sin\left(-\frac{3\pi}{2}+x\right)}{1+\cos\left(-\frac{\pi}{2}+x\right)}\right].$$

		\defword{解析. }诱导公式, 两倍角公式, 半角公式.
			\begin{align*}
				\YS &= \frac{1+\sin x}{\cos x} \cdot \left[\frac{2 \sin x \cos x}{1+\cos\left(\frac{\pi}{2}-x\right)}-\frac{\cos x}{1+\sin x}\right]\\
				&= \frac{1+\sin x}{\cos x} \cdot \left[\frac{2 \sin x \cos x}{1+\sin x}-\frac{\cos x}{1+\sin x}\right]\\
				&= \answer{$2\sin x-1$}.
			\end{align*}
		~\\

		\defword{12. }已知$\alpha \in \left(0, \dfrac{\pi}{2}\right)$有$3\sin \alpha - 4\cos \alpha = 1$, 求$\tan \alpha$.

		\defword{解析. }辅助角公式, 同角三角比, 和角公式.
			\begin{align*}
				3\sin \alpha - 4\cos \alpha = 5\sin \left(\alpha - \arctan\frac{4}{3}\right)=1 &\Rightarrow \sin\left(\alpha - \arctan\frac{4}{3}\right)=\dfrac{1}{5}, \alpha-\arctan\dfrac{4}{3}\in\left(-\frac{\pi}{2}, \frac{\pi}{2}\right)\\
				&\Rightarrow \tan \alpha = \frac{\tan\left(\alpha - \arctan\frac{4}{3}\right) \cdot \frac{4}{3}}{1-\tan\left(\alpha - \arctan\frac{4}{3}\right)\cdot\frac{4}{3}}=\answer{$\frac{6+\sqrt{6}}{4}$}.
			\end{align*}
		~\\

		\defword{13. }已知$\cos\alpha \in \left[\dfrac{1}{2}, 1\right)$, 求$\tan \dfrac{\alpha}{2}(\sin \alpha + \tan \alpha)$的最大值.

		\defword{解析. }半角公式, 同角三角比, 函数在区间上的极值.
			\begin{align*}
				\max_{\cos\alpha \in \left[\frac{1}{2}, 1\right)} \tan \dfrac{\alpha}{2}(\sin \alpha + \tan \alpha) &= \max_{\cos\alpha \in \left[\frac{1}{2}, 1\right)} \frac{1-\cos \alpha}{\sin \alpha} \left(\sin \alpha + \frac{\sin \alpha}{\cos \alpha}\right)\\
				&= \max_{\cos\alpha \in \left[\frac{1}{2}, 1\right)} 1-\cos\alpha + \frac{1-\cos\alpha}{\cos\alpha}\\
				&= \max_{\cos\alpha \in \left[\frac{1}{2}, 1\right)} \frac{1}{\cos \alpha} - \cos \alpha\\
				&= \at{\frac{1}{\cos \alpha}-\cos \alpha}{\cos \alpha = \frac{1}{2}}\\
				&= \answer{$\frac{3}{2}$}.
			\end{align*}
		~\\

		\defword{14. }利用两倍角公式及三倍角公式$\sin 3\theta = 3\sin \theta - 4\sin^3 \theta, \cos 3\theta = 4\cos^3 \theta - 3\cos \theta$求$\sin 18\degree$.

		\defword{解析. }同角三角比, 两倍角公式, 三倍角公式.

			考虑$\cos 54\degree$ ($90\degree - 3\times 18\degree = 2\times 18\degree$), 有

			$$\cos 54\degree = 4\cos^3 18\degree - 3\cos 18\degree = \sin 36\degree = 2\sin 18\degree \cos 18\degree$$

			即

			$$4\cos^3 18\degree - 3\cos 18\degree = 2\sin 18\degree \cos 18\degree.$$

			两边同时除以$\cos 18\degree$, 带入$\cos 18\degree = \sqrt{1 - \sin^2 18\degree}$, 有

			$$4(1-\sin^2 18\degree)-3=2\sin 18\degree$$

			即

			$$4\sin^2 18\degree + 2\sin 18\degree - 1 = 0$$

			有

			$$\sin 18\degree = \frac{-1\pm\sqrt{5}}{4} \text{(舍负)}.$$

			综上, \answer{$\sin 18\degree = \dfrac{-1+\sqrt{5}}{4}$}.

	\section{附加题}
		\defword{15. } 已知$\dfrac{\sin \alpha}{\sin \beta}=p, \dfrac{\cos \alpha}{\cos \beta}=q, \abs{p} \neq 1, q\neq 0, $用$p, q$表达$\tan \alpha \cdot \tan \beta$.

		\defword{解析. }同角三角比.

		由已知, 显然有

		$$\sin \alpha = p\sin \beta \eqno{(1)}$$

		和

		$$\cos \alpha = q \cos \beta \eqno{(2)}.$$

		考虑$(1)/(2)$得

		$$\tan \alpha = \frac{p}{q} \tan \beta.$$

		两边同时乘以$\tan \beta$得

		$$\tan \alpha \tan \beta = \frac{p}{q} \tan^2 \beta.$$

		考虑$(1)^2 + (2)^2$得

		$$\sin^2 \alpha + \cos^2 \alpha = 1 = p^2 \sin^2 \beta + q^2 \cos^2 \beta,$$

		即

		$$\sin^2 \beta (p^2 - 1)=\cos^2 (1 - q^2),$$

		即

		$$\tan^2 \beta = \frac{1-q^2}{p^2-1}.$$

		故\answer{$\YS=\dfrac{p(1-q^2)}{q(p^2-1)}$}.

	\section{特别致谢}
		\textbf{\textcolor{allangreen}{MathxStudio/LaTeX-Templates}}: 提供排版模版的模版.

\end{document}