%!TEX TX-program = xelatex
\documentclass[8pt]{article}
\usepackage{allan-eason}

\usetikzlibrary{positioning}
\usetikzlibrary{svg.path}

\graphicspath{ {./images/} }


\author{\normalfont\sffamily\large\bfseries{高一(4)班\ 邵亦成\ 48号}}
\title{\normalfont\sffamily\huge\bfseries{\textcolor{allanblue}{220323}\ \textcolor{allancyan}{三角恒等式与正余弦定理}\ 题目选解}}
\date{}

\geometry{a4paper, scale=0.8}

\lhead{\textcolor{allanblue}{220323}\ \textcolor{allancyan}{三角恒等式与正余弦定理}\ 题目选解}

\begin{document}

	\maketitle

	\section{填空题}
		\defword{1. }若$\tan \left(\pi + \alpha \right)=2$, 则$\sin 2\alpha=$\answord{$\dfrac{4}{5}$}.

		\answord{解析. }诱导公式, 两倍角公式; 计算器的使用.

		\calword{计算器. }$\sin\left[2\left(\arctan 2 - 180\right)\right]$.
		~\\

		\defword{2. }在$\triangle ABC$中, $\sin A : \sin B : \sin C = 4:5:6$, 则最大内角的余弦值为\answord{$\dfrac{1}{8}$}.

		\answord{解析. }正弦定理, 余弦定理.
		~\\

		\defword{3. }等腰$\triangle ABC$的底边长为$2$, 面积为$3$, 则其外接圆半径为\answord{$\dfrac{5}{3}$}.

		\answord{解析. }正弦定理.
		~\\

		\defword{4. }在$\triangle ABC$中, $a=3\sqrt{2}, b=3\sqrt{3}, B=\dfrac{\pi}{3}$, 则$A=$\answord{$\dfrac{\pi}{4}$}.

		\answord{解析. }正弦定理.
		~\\

		\defword{5. } $\alpha$为第四象限角, $\tan \alpha = -\dfrac{7}{24}$, 则$\tan \dfrac{\alpha}{2}=$\answord{$-\dfrac{1}{7}$}.

		\answord{解析. }半角公式; 计算器的使用.

		\calword{计算器. }$\tan \left[\dfrac{\arctan \left(-\dfrac{7}{24}\right)}{2}\right]$.
		~\\

		\defword{6. } 若$\dfrac{1-\cos x}{\sin x}=3$, 则$\cos x + \cot \dfrac{x}{2}=$\answord{$-\dfrac{7}{15}$}.

		\answord{解析. }同角三角比, 半角公式; 计算器的使用.
		~\\

		\defword{7. } 在$\triangle ABC$中, $\sin (A-B) + \sin C=\sin 2B$, $C=40\degree$, 则$B=$\answord{$70\degree \lgor 90\degree$}.

		\answord{解析. }差角公式, 两倍角公式; 计算器的使用.

		\calword{计算器. }$\mathrm{SOLVE} \sin(140-2x)+\sin(40)=\sin(2x)$.
		~\\

		\defword{8. } 在$\triangle ABC$中$b=2a\cos C$, 则当$3\tan \dfrac{B}{2} + \cot \dfrac{B}{2}$取得最小值时, 三角形的形状为\answord{等边三角形}.

		\answord{解析. }同角三角比, 两倍角公式. 基本不等式.
		~\\

		\defword{9. } 使$\triangle ABC$有唯一解的有 \answord{(1), (4)}.
			\begin{enumerate}[label=(\arabic*)]
				\item $a=8, c=6, B=131\degree$.
				\item $A=12\degree, \sin B=\dfrac{\sqrt{2}}{2}, a=1$.
				\item $a=7, b=12, A=\dfrac{\pi}{6}$.
				\item $A=95\degree, a=24, c=21$.
			\end{enumerate}

		\answord{解析. }正余弦定理.
		~\\

		\defword{10. }设$f(\tan \alpha)=\cos 2\alpha, g(\tan \alpha)=\tan \left(\dfrac{\pi}{4} - \alpha \right).$ 当$x\neq \dfrac{k\pi}{2} (k\in \ZZ)$, $5f(\cos x)=g(\cos 2x)$的解集为\answord{$\displaystyle \bigcup_{k\in \ZZ} \left\{\frac{\pi}{3}+k\pi, \frac{2\pi}{3}+k\pi\right\}$}.

		\answord{解析. } 将映射$f(x)$化简为有理多项式的形式. 注意到正切半角公式

		$$\cos \alpha = \frac{1-\tan^2 \frac{1}{2} \alpha}{1+\tan^2 \frac{1}{2} \alpha},$$

		带入$\alpha = 2\alpha$即可知

		$$f(x)=\frac{1-x^2}{1+x^2}.$$

		将$g(x)$化简为有理多项式的形式. 考虑差角公式显然有

		$$g(x)=\frac{1-x}{1+x}.$$

		原方程即可化为

		$$5 \frac{1 - \cos^2 x}{1 + \cos^2 x} = \frac{1 - \cos 2x}{1 + \cos 2x}.$$

		将两倍角公式代入, 考虑同角三角比的关系, 有

		$$\cos x = \pm \frac{1}{2},$$

		即

		$$x \in \bigcup_{k\in \ZZ} \left\{\frac{\pi}{3}+k\pi, \frac{2\pi}{3}+k\pi\right\}.$$

	\section{解答题}
		\defword{11. }以$a=10, A=\dfrac{\pi}{3}$和一个给定的$b$作为一组条件, 当该三角形

			\begin{enumerate}[label=(\arabic*)]
				\item 有唯一解;
				\item 有两解;
				\item 无解;
			\end{enumerate}

			求$b$的取值范围.

		\answord{解析. }余弦定理. 二次函数的零点.

			由余弦定理显然有

			$$a^2=b^2+c^2-2bc\cos A.$$

			整理, 得

			$$c^2-bc+b^2-100=0, \Delta = b^2 - 4b^2 + 400 = -3b^2 + 400.$$

			二次函数$f(c) = c^2-bc+b^2-100$的对称轴为直线$x=\dfrac{b}{2} > 0$, $f(0) = b^2-100$.

			\begin{enumerate}[label=(\arabic*)]
				\item 有唯一解 $\Leftrightarrow \Delta = 0 \lgor f(0) \leq 0 \Iff \ansmath{b\in \left\{\dfrac{20\sqrt{3}}{3}\right\} \cup (0, 10]}$.
				\item 有两解 $\Leftrightarrow \Delta > 0 \lgand f(0) > 0 \Iff \ansmath{b \in \left(10, \dfrac{20\sqrt{3}}{3}\right)}$.
				\item 无解 $\Leftrightarrow \Delta < 0 \Iff \ansmath{b\in \left(\dfrac{20\sqrt{3}}{3}, +\infty\right)}$.
			\end{enumerate}
		~\\

		\defword{12. }若关于$x$的方程$\sin 2x+\cos 2x-2\sin^2 x=\left(\dfrac{1}{5}\right)^a - 1$有解, 求$a$的取值范围.

		\answord{解析. }两倍角公式. 可化简为代数不等式的超越不等式.

			整理, 原方程有解当且仅当方程

			$$\sin 2x+2\cos 2x=\left(\frac{1}{5}\right)^a$$

			有解,

			\begin{align*}
				\sum_{\sin 2x+2\cos 2x=\left(\frac{1}{5}\right)^a} > 0 &\Rightarrow \left(\frac{1}{5}\right)^a \in \at{R(f)}{f(x)=\sin 2x+\cos 2x} = \at{R(g)}{g(x)=\sin x + 2\cos x = \sqrt{5} \sin(x + \arctan 2)}\\
				&\Rightarrow \left(\frac{1}{5}\right)^a \in \left[-\sqrt{5}, \sqrt{5}\right]\\
				&\Rightarrow \ansmath{a \in \left[-\frac{1}{2}, +\infty \right)}.
			\end{align*}
		~\\

		\defword{13. }在$\triangle ABC$中, $a=\sqrt{7}, b=3, \sin A + \sqrt{7} \sin B = 2\sqrt{3},$ 求$A$的大小. 若$\triangle ABC$为钝角三角形, 求$S_{\triangle ABC}$.

		\answord{解析. }正余弦定理, 三角形面积公式.

			\begin{align*}
				\triangle ABC &\Rightarrow \frac{a}{\sin A}=\frac{b}{\sin B}\\
				&\Rightarrow \sqrt{7} \sin B = 3\sin A\\
				&\Rightarrow \sin A = \frac{\sqrt{3}}{2}\\
				&\Rightarrow A\in\left\{\frac{\pi}{3}, \frac{2\pi}{3}\right\}.
			\end{align*}

			注意到$a=\sqrt{7}<b=3$, 由大边对大角有$\ansmath{A=\dfrac{\pi}{3}}$.

			\begin{align*}
				\triangle ABC &\Rightarrow a^2 = b^2 + c^2 - 2bc \cos A\\
				&\Rightarrow c^2 - 3c + 2 =0\\
				&\Rightarrow c\in\{1, 2\}.
			\end{align*}

			考虑$c=1, \cos B = -\dfrac{1}{2\sqrt{7}}<0$符合题意.

			考虑$c=2, \cos B = \dfrac{1}{2\sqrt{7}}>0$不符题意, 舍去.

			于是$c=1$,

			\begin{align*}
				S_{\triangle ABC} &= \frac{1}{2} bc\sin A\\
				&= \frac{1}{2} \times 3 \times 1 \times \frac{\sqrt{3}}{2}\\
				&= \ansmath{\frac{3\sqrt{3}}{4}}.
			\end{align*}
		~\\

		\defword{14. }$\triangle ABC$中$c=\frac{\sqrt{2}}{2}$, 且$S_{\triangle ABC}=\frac{\sqrt{2}}{4} ab\sin 2C$, 求该三角形面积的最大值.

		\answord{解析. }三角形面积公式, 半角公式, 诱导公式, 和差化积公式. 基本不等式.

			\begin{align*}
				S_{\triangle ABC} = \frac{1}{2} ab \sin C = \frac{\sqrt{2}}{4}ab\sin 2C &\Rightarrow 2\sin C = \sqrt{2} \sin 2C = 2\sqrt{2} \sin C \cos C\\
				&\Rightarrow \cos C = \frac{\sqrt{2}}{2}\\
				&\Rightarrow C = \frac{\pi}{4}.
			\end{align*}

			由正弦定理及条件有

			$$\frac{a}{\sin A}=\frac{b}{\sin B}=\frac{c}{\sin C}=1 \Rightarrow a=\sin A, b=\sin B.$$

			\begin{align*}
				\max S_{\triangle ABC} &= \max \left(\frac{1}{2} ab\sin C\right)\\
				&= \frac{\sqrt{2}}{4} \max (ab)\\
				&= \frac{\sqrt{2}}{4} \max (\sin A \sin B)\\
				&\leq \frac{\sqrt{2}}{4} \frac{(\sin A + \sin B)^2}{4} \tag*{$\Iff A=B \lgor A+B=\pi$}\\
				&= \frac{\sqrt{2}}{16} \left(2 \sin \frac{A+B}{2} \cos \frac{A-B}{2}\right)^2\\
				&= \frac{\sqrt{2}}{4} \left(\sin \frac{A+B}{2} \cos \frac{A-B}{2}\right)^2\\
				&\leq \frac{\sqrt{2}}{4} \left(\sin \frac{A+B}{2}\right)^2 \tag*{$\Iff A=B$}\\
				&= \frac{\sqrt{2}}{4} \frac{1-\cos(A+B)}{2}\\
				&= \frac{\sqrt{2}}{8} (1+\cos C)\\
				&= \frac{\sqrt{2}}{8} \left(1 + \frac{\sqrt{2}}{2}\right)\\
				&= \ansmath{\frac{\sqrt{2} + 1}{8}}.
			\end{align*}

	\section{附加题}
		\defword{15. }若$\displaystyle \sqrt{2} \cos \frac{\alpha}{2} - \sin \frac{\alpha}{2} \sec \beta + \cos \frac{\alpha}{2} \sec \beta = 0, \sqrt{2} \sin \frac{\alpha}{2} + \cos \frac{\alpha}{2} \tan \beta - \sin \frac{\alpha}{2} \tan \beta = 0$, 求$\sin \alpha$.

		\answord{解析. }同角三角比.

			$$\sqrt{2} \cos \frac{\alpha}{2} - \sin \frac{\alpha}{2} \sec \beta + \cos \frac{\alpha}{2} \sec \beta = \sqrt{2} \cos \frac{\alpha}{2} + \sec \beta \left(\cos \frac{\alpha}{2} - \sin \frac{\alpha}{2}\right)=0,$$

			$$\sqrt{2} \sin \frac{\alpha}{2} + \cos \frac{\alpha}{2} \tan \beta - \sin \frac{\alpha}{2} \tan \beta = \sqrt{2} \sin \frac{\alpha}{2} + \tan \beta \left(\cos \frac{\alpha}{2} - \sin \frac{\alpha}{2}\right)=0,$$

			相减, 有

			$$\left(\cos \frac{\alpha}{2} - \sin \frac{\alpha}{2}\right)\left(\sqrt{2} + \sec \beta - \tan \beta\right) = 0,$$

			故有

			$$\tan \beta - \sec \beta = \sqrt{2} \lgor \sin \frac{\alpha}{2} = \cos \frac{\alpha}{2}.$$

			考虑$\sin \frac{\alpha}{2} = \cos \frac{\alpha}{2}$, 等式成立 $\displaystyle \Iff \cos \frac{\alpha}{2} = \sin \frac{\alpha}{2} = 0 \Rightarrow \alpha \in \emptyset$.

			故有$\tan \beta - \sec \beta = \sqrt{2}$. 又考虑到$\tan^2 \beta - \sec^2 \beta = (\tan \beta + \sec \beta)(\tan \beta - \sec \beta) = -1,$ 可得出

			$$\tan \beta + \sec \beta = -\dfrac{\sqrt{2}}{2},$$

			故有

			$$\tan \beta = \frac{\sqrt{2}}{4}, \sec \beta = -\frac{3\sqrt{2}}{4}.$$

			代入, 有$3\sin \dfrac{\alpha}{2} + \cos \dfrac{\alpha}{2}=0$, 解得

			$$\sin \frac{\alpha}{2} = \pm \frac{\sqrt{10}}{10}, \cos \frac{\alpha}{2} = \mp \frac{\sqrt{10}}{10},$$

			于是

			$$\sin \alpha = 2 \sin \frac{\alpha}{2} \cos \frac{\alpha}{2} = - \frac{3}{5}.$$

	\section{特别致谢}
		\textbf{\textcolor{allangreen}{MathxStudio/LaTeX-Templates}}: 提供排版模版的模版.

\end{document}