%!TEX TX-program = xelatex
\documentclass[8pt]{article}

\usepackage[UTF8]{ctex}
\usepackage{graphicx}
\usepackage{enumerate}
\usepackage{geometry}
\usepackage{amsmath}
\usepackage{amssymb}
\usepackage{amsfonts}
\newtheorem{theorem}{Theorem}
\newtheorem{lemma}{Lemma}
\newtheorem{proof}{Proof}

\author{高一(6)班\ 邵亦成\ 26号}
\title{3 集合测验''(难题)}
\date{2021年09月22日}

\geometry{a4paper, scale=0.8}

\begin{document}

	\maketitle

	\begin{enumerate}

		\item
			有限集$S$的全部元素的乘积,称为这个数集$S$的积数,例如集合$\{2\}$的积数是$2$,集合$\{2,3\}$的积数是$6$.

			\begin{enumerate}[ (1) ]

				\item 求集合$M=\left\{\frac{1}{2},\frac{1}{3},\frac{1}{4},\frac{1}{5}\right\}$的所有非空子集的积数之和.

				\item 求集合$M=\left\{x|x=\frac{1}{n},2\leq n\leq100, n\in\mathbb{N}\right\}$的所有非空子集的积数之和.

			\end{enumerate}

			~\\

			下证\textbf{引理}:集合$M$的积数$=$$$\prod_{x\in M}(1+x)-1.$$

			令$|M|=n$,构造方程

			$$f(x)=\prod_{k\in M}(x-k)=0.$$

			展开,得

			$$x^n+a_1x^{n-1}+a_2x^{n-2}+\cdots+a_{n-1}x+a_n=0.$$

			由高次方程的Vieta定理,有:

			$$
			\left\{
			\begin{array}{rcl}
			\sum_{1\leq i \leq n}x_i&=&-a_1\\
			\sum_{1\leq i < j \leq n}x_ix_j&=&a_2\\
			&\vdots&\\
			\prod_{1\leq i \leq n}x_i&=&(-1)^{n}a_n\\
			\end{array}
			\right..
			$$

			易得,所求式$=-a_1+a_2-\cdots+(-1)^{n}a_n$.

			若$n$是偶数,所求式$=f(-1)-1$;若$n$是奇数,所求式$=-f(-1)-1$.

			代入化简,得集合$M$的积数$=g(M)=$$$\prod_{x\in M}(1+x)-1.$$

			将$M=\left\{\frac{1}{2},\frac{1}{3},\frac{1}{4},\frac{1}{5}\right\}$代入得$g(M)=2$.

			将$M=\left\{x|x=\frac{1}{n},2\leq n\leq100, n\in\mathbb{N}\right\}$代入得$g(M)=\frac{99}{2}$.

	\end{enumerate}

\end{document}