%!TEX TX-program = xelatex
\documentclass[8pt]{article}

\usepackage[UTF8]{ctex}
\usepackage{graphicx}
\usepackage{enumerate}
\usepackage{geometry}
\usepackage{amsmath}
\usepackage{amssymb}
\usepackage{amsfonts}

\author{高一(6)班\ 邵亦成\ 26号}
\title{1 集合测验(难题)}
\date{2021年09月08日}

\geometry{a4paper, scale=0.8}

\begin{document}

	\maketitle

	\begin{enumerate}

		\item
			设$\left[m\right]$表示不超过实数$m$的最大整数,则集合$\left\{x\in\mathbb{R}|9x^2-30\left[x\right]+20=0\right\}$中所有元素的和为?.

			\begin{enumerate}

				\item
					思路:
						\begin{enumerate}

							\item
								限定$\left[x\right]$的范围($[x]$是有限个,方便求解).

							\item
								代入原式进行求解.

						\end{enumerate}

				\item
					解答:

					将$[x]\leq x$代入原方程得:

					$$9x^2+20=30[x]\leq30x.$$

					解不等式,得$x\in \left[\frac{5-\sqrt{5}}{3},\frac{5+\sqrt{5}}{3}\right]$.

					$\frac{5-\sqrt{5}}{3}\approx0.92, \frac{5+\sqrt{5}}{3}\approx2.41 \Rightarrow [x] \in \{0,1,2\}$.

					分类讨论:

					\begin{enumerate} [ $1^{\circ}$ ]

						\item
							$[x]=0 \Rightarrow x\in[0,1)$
							代入得:
							$$9x^2+20=0$$
							无实数解.

						\item
							$[x]=1 \Rightarrow x\in[1,2)$
							代入得:
							$$9x^2-10=0$$
							解得$x_1=\sqrt{\frac{10}{9}},x_2=-\sqrt{\frac{10}{9}}$.

							$\because x\in[1,2) \therefore x=\sqrt{\frac{10}{9}}$.

						\item
							$[x]=2 \Rightarrow x\in[2,3)$
							代入得:
							$$9x^2-40=0$$
							解得$x_1=\sqrt{\frac{40}{9}},x_2=-\sqrt{\frac{40}{9}}$

							$\because x\in[2,3) \therefore x=\sqrt{\frac{40}{9}}$.

					\end{enumerate}

					综上所述,$x=\sqrt{\frac{10}{9}}$或$x=\sqrt{\frac{40}{9}}$,$\sum{x}=\sqrt{10}$.

			\end{enumerate}

		\item
			设$S$为有限集合,$A_1, A_2, \cdots A_{2019}$为$S$的子集,$|X|$表示集合$X$中元素的个数. 已知对每个正整数$i\in[1,2019]$都有$\left|A_i\right|\geq\frac{1}{5}|S|$.若对于任意集合$S$总是存在$x\in S$在至少$k$个集合$A_i$中出现,则$k$的最大值是多少?并加以证明.

			$k_{\max}=404.$

			下证:$k_{\max} \geq 404$.

			定义元素$t$在$S$的子集$A_i$出现的次数为$f_t$($t\in S, i \in [1,2019] \cap \mathbb{Z}$).

			易证

			$$\sum_{t\in S}{f_t} = \sum_{i=1}^{2019}{\left|A_i\right|}\geq\frac{2019}{5}|S|=403.8|S|.$$

			于是

			$$\exists t \in S: f_t \geq \lceil 403.8 \rceil = 404.$$

			于是$k_{\max} \geq 404$.

			下证:$k \ngeq 405$.

			构造
			$$S=\{1,2,3,4,5\},$$
			$$A_1=A_2=\cdots A_{404}=\{1\},$$
			$$A_{405}=A_{406}=\cdots A_{808}=\{2\},$$
			$$A_{809}=A_{810}=\cdots A_{1212}=\{3\},$$
			$$A_{1213}=A_{1214}=\cdots A_{1616}=\{4\},$$
			$$A_{1617}=A_{1618}=\cdots A_{2019}=\{5\}.$$

			此时, $k=404$.

			综上所述, $404\leq k_{\max} < 405$, $k_{\max}=404$.

	\end{enumerate}

\end{document}