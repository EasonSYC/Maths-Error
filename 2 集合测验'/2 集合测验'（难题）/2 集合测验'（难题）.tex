%!TEX TX-program = xelatex
\documentclass[8pt]{article}

\usepackage[UTF8]{ctex}
\usepackage{graphicx}
\usepackage{enumerate}
\usepackage{geometry}
\usepackage{amsmath}
\usepackage{amssymb}
\usepackage{amsfonts}

\author{高一(6)班\ 邵亦成\ 26号}
\title{2 集合测验'(难题)}
\date{2021年09月15日}

\geometry{a4paper, scale=0.8}

\begin{document}

	\maketitle

	\begin{enumerate}

		\item
			集合$M=\left\{66,-14,2333,10,911,-1,0,\pi,3.1415,\sqrt{7}\right\}$有$10$个元素,设$M$的所有非空子集为$M_i \left(i\leq1023, i\in\mathbb{N}^{*}\right)$,每一个$M_i$中所有元素的乘积为$m_i$,则$\sum_{i=1}^{1023}{m_i}=$?.

			~\\

			考虑构造方程

			$$\prod_{i\in M}{(x-i)}=0.$$

			易得,此方程的解集合为集合$M$.

			展开,得

			$$a_1x^{10}+a_2x^9+a_3x^8+\cdots+a_{10}x+a_{11}=0$$

			由高次方程的$Vi\grave{e}te$定理有:

			$$
			\left\{
				\begin{aligned}
					\sum_{i=1}^{10}{x_i}&=&-\frac{a_2}{a_1}\\
					\sum_{i=1}^{10}{\sum_{j=1, j\neq i}^{10}{x_ix_j}}&=&\frac{a_3}{a_1}\\
					&\vdots&\\
					\prod_{i=1}^{10}{x_i}&=&\frac{a_{11}}{a_1}\\
				\end{aligned}
			\right.
			$$

			显然,所求内容即为$\frac{-a_2+a_3-a_4+a_5-\cdots-a_{10}+a_{11}}{a_1}$

			易得,$a_1=1$,所求内容即为$-a_2+a_3-a_4+a_5-\cdots-a_{10}+a_{11}$.

			将$x=-1$代入展开后的方程左边,即为$a_1-a_2+a_3-a_4+a_5-\cdots-a_{10}+a_{11}$.

			由于$x=-1\in M$,等式右边即为0,即所求内容为$-1$.

		~\\

		\item
			称正整数集合$A=\{a_1,a_2,\cdots,a_n\}(1\leq a_1 < a_2 < \cdots < a_n, n \geq 2)$具有性质$P$,若它满足$\forall 1\leq i\leq j\leq n: a_ia_j \in A \vee \frac{a_j}{a_i}\in A$.

			\begin{enumerate}[ (1) ]

				\item 分别判断集合$\{1,3,6\}$与$\{1,3,4,12\}$是否具有性质$P$.

				(过程略),$\{1,3,6\}$不具有,$\{1,3,4,12\}$具有.

				\item 设正整数集合$A=\{a_1,a_2,\cdots,a_n\}(1\leq a_1 < a_2 < \cdots < a_n, n \geq 2)$具有性质$P$,证明:$\forall i \in \mathbb{N}^{*}, i\leq n, a_i | a_n$.

				反证法:假设$\exists i \in \mathbb{N}^{*}, i\leq n, a_i \nmid a_n$.

				可知$a_i\neq 1$.

				考虑$a_i a_n > a_n$与$a_n$最大性矛盾.

				考虑$\frac{a_n}{a_i}\notin \mathbb{N}^{*}$于是$\frac{a_n}{a_i}\notin A$.

				与集合$A$具有性质$P$矛盾.

				于是$\forall i \in \mathbb{N}^{*}, i\leq n, a_i | a_n$.

			\end{enumerate}

	\end{enumerate}

\end{document}