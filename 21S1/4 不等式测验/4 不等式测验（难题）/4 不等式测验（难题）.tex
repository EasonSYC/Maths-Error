%!TEX TX-program = xelatex
\documentclass[8pt]{article}

\usepackage[UTF8]{ctex}
\usepackage{graphicx}
\usepackage{enumerate}
\usepackage{geometry}
\usepackage{amsmath}
\usepackage{amssymb}
\usepackage{amsfonts}
\newtheorem{theorem}{Theorem}
\newtheorem{lemma}{Lemma}
\newtheorem{proof}{Proof}

\author{高一(6)班\ 邵亦成\ 26号}
\title{4 不等式测验(难题)}
\date{2021年09月29日}

\geometry{a4paper, scale=0.8}

\begin{document}

	\maketitle

	\begin{enumerate}

		\item
			解关于$x$的不等式$[(a-1)x-(a-2)](x-2)>0.\ (a\in \mathbb{R})$

			~\\

			\begin{enumerate} [$1^\circ$]
				\item
					$a=1$时,

					原不等式可化简为$x-2>0 \Rightarrow x>2$即$x\in(2,+\infty)$.

				\item
					$a\neq 1$时,

					有:方程$[(a-1)x-(a-2)](x-2)=0$的两根为$x_1=\frac{a-2}{a-1}=1+\frac{1}{1-a}, x_2=2.$

					\begin{enumerate} [(i)]
						\item
							$a>1$时,

							恒有$x_1<x_2$,

							于是有$x\in \left(-\infty, 1+\frac{1}{1-a}\right)\cup(2,+\infty)$.

						\item
							$a<1$时,

							原不等式可化为$[(1-a)x-(2-a)](x-2)<0$.

							\begin{enumerate} [a.]
								\item
									$x_1<x_2$即$a<0$时,$x\in\left(1+\frac{1}{1-a}, 2\right)$.

								\item
									$x_1=x_2$即$a=0$时,$x\in\emptyset$.

								\item
									$x_1>x_2$即$0<a<1$时,$x\in\left(2, 1+\frac{1}{1-a}\right)$.
							\end{enumerate}

					\end{enumerate}

			\end{enumerate}


		综上所述,原不等式的解集$=$

		$$
		\left\{
		\begin{array}{rcl}

			\left(1+\frac{1}{1-a}, 2\right)&,&a\in(-\infty, 0)\\
			\emptyset&,&a=0\\
			\left(2, 1+\frac{1}{1-a}\right)&,&a\in(0, 1)\\
			\left(2, +\infty \right)&,&a=1\\
			\left(-\infty, 1+\frac{1}{1-a}\right)\cup(2,+\infty)&,&a\in(0, +\infty)\\

		\end{array}
		\right..
		$$

		这种写法是错误的,综上应该按照如下格式:

		综上:当$a\in(-\infty, 0)$时,原不等式解集为$\left(1+\frac{1}{1-a}, 2\right)$;当$a=0$时,原不等式解集为$\emptyset$;当$a\in(0, 1)$时,原不等式解集为$\left(2, 1+\frac{1}{1-a}\right)$;当$a=1$时,原不等式解集为$\left(2, +\infty \right)$;当$a\in(0, +\infty)$时,原不等式解集为$\left(-\infty, 1+\frac{1}{1-a}\right)\cup(2,+\infty)$.

	\end{enumerate}

\end{document}