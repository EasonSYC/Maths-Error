%!TEX TX-program = xelatex
\documentclass[8pt]{article}

\usepackage{ctex}
\usepackage{graphicx}
\usepackage{enumitem}
\usepackage{geometry}
\usepackage{amsmath}
\usepackage{amssymb}
\usepackage{amsfonts}
\usepackage{tikz}
\usepackage{extarrows}
\usetikzlibrary{positioning}
\usetikzlibrary{svg.path}
\usepackage{xcolor}
\usepackage{soul}

\graphicspath{ {./images/} }

\author{高一(6)班\ 邵亦成\ 26号}
\title{7 幂指对测验(简单题)}
\date{2021年10月27日}

\geometry{a4paper, scale=0.8}

\begin{document}

	\maketitle

	\begin{enumerate}[label=(\arabic*)]
		\item 不等式组$(2-x)(x-6)\leq 0$的解集为 \st{$(-\infty , 2]\cup [0, +\infty)$} \textcolor{red}{$(-\infty, 2]\cup[6, +\infty)$}.
			~\\

			\st{$\text{我不是很明白为什么一个不等式也是不等式组}$}.	

			不多作评价, 能把6抄到卷子上抄成0的可能也只有我了吧.

		\item 不等式$\displaystyle \left(\frac{1}{x}+2\right)\cdot\frac{1}{x}<0$的解集为 \st{$\displaystyle \left(-\frac{1}{2}, 0\right)$} \textcolor{red}{$\displaystyle \left(-\infty, -\frac{1}{2}\right)$}.
			~\\

			$\displaystyle \left(\frac{1}{x}+2\right)\cdot\frac{1}{x}<0 \Rightarrow \frac{1}{x} \in (-2, 0) \Rightarrow x \in \left(-\infty, -\frac{1}{2}\right).$

		\item 关于$x$的不等式$x^2+mx+m^2+3m<0$的解集包含区间$(1, 2)$, 则$m$的取值范围是 \st{$[-2-\sqrt{3}, -2+\sqrt{3}]$} \textcolor{red}{$[-2-\sqrt{3}, -1]$}.
			~\\

			不多作评价, 我也不知道我为什么会觉得$-1$比$-2+\sqrt{3}$大.

			$$
			\begin{array}{cl}
				&\left\{
				\begin{array}{rcl}
					1+m+m^2+3m&\leq&0\\
					4+2m+m^2+3m&\leq&0\\
				\end{array}
				\right.\\
				\Rightarrow&
				\left\{
				\begin{array}{rcl}
					m&\in&[-2-\sqrt{3}, -2+\sqrt{3}]\\
					m&\in&[-4, -1]\\
				\end{array}
				\right.\\
				\Rightarrow&
				m\in[-2-\sqrt{3}, -2+\sqrt{3}]\cap[-4, -1]=[-2-\sqrt{3}, -1].
			\end{array}\\
			$$

	\end{enumerate}

\end{document}