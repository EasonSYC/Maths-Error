%!TEX TX-program = xelatex
\documentclass[8pt]{article}

\usepackage{ctex}
\usepackage{graphicx}
\usepackage{enumitem}
\usepackage{geometry}
\usepackage{amsmath}
\usepackage{amssymb}
\usepackage{amsfonts}
\usepackage{tikz}
\usepackage{extarrows}
\usetikzlibrary{positioning}
\usetikzlibrary{svg.path}
\usepackage{xcolor}
\usepackage{soul}

\graphicspath{ {./images/} }

\author{高一(6)班\ 邵亦成\ 26号}
\title{7 幂指对测验(难题)}
\date{2021年10月27日}

\geometry{a4paper, scale=0.8}

\begin{document}

	\maketitle

	\begin{enumerate}[label=(\arabic*)]
		\item 已知$a>1, b>1$, 求$\displaystyle \frac{b^2}{a-1} + \frac{a^2}{b-1}$的最小值.
			~\\

			我也不知道为什么我最后一步就是卡住了.

			\textbf{法一 (基本不等式\ 轮换求和)}: 

				$$
				\begin{array}{rcl}
				a>1, b>1&\Rightarrow&a-1>0, b-1>0\\
				&\Rightarrow&\displaystyle \frac{b^2}{a-1}+4(a-1)\geq 4b, \frac{a^2}{b-1}+4(b-1)\geq 4a\\
				&\Rightarrow&\displaystyle \frac{b^2}{a-1}+4(a-1) + \frac{a^2}{b-1}+4(b-1)\geq 4a + 4b\\
				&\Rightarrow&\displaystyle \frac{b^2}{a-1}+\frac{a^2}{b-1}\geq 8,
				\end{array}
				$$

				等号成立当且仅当$\displaystyle \frac{b^2}{a-1}=4(a-1), \frac{a^2}{b-1}=4(b-1)$即$a=b=2$.
			~\\

			\textbf{法二 (基本不等式\ 换元)}: 令$a-1=m, b-1=n (m>0, n>0)$, 则有: 

				$$
				\begin{array}{rcl}
					\displaystyle \frac{b^2}{a-1} + \frac{a^2}{b-1}&=&\displaystyle \frac{(n+1)^2}{m}+\frac{(m+1)^2}{n}\\
					&=&\displaystyle \frac{n^3+2n^2+n+m^3+2m^2+m}{mn}\\\\
					&=&\displaystyle \frac{(m+n)(m^2+n^2-mn)+2(m^2+n^2)+m+n}{mn}\\\\
					&\geq&\displaystyle \frac{2\sqrt{mn}(2mn-mn)+2\times 2mn+2\sqrt{mn}}{mn}\\\\
					&=&2\displaystyle \sqrt{mn}+4+\frac{8}{\sqrt{mn}}\\
					&\geq&8,
				\end{array}
				$$

				等号成立当且仅当

				$$
				\left\{
				\begin{array}{rcl}
					m&=&n\\
					2\sqrt{mn}&=&\displaystyle\frac{8}{\sqrt{mn}}\\
				\end{array}
				\right.
				$$

				即$m=n=2$即$a=b=1$.
			~\\

			\textbf{法三 (基本不等式\ 暴力)}:

				$$
				\begin{array}{rcl}
					\displaystyle \frac{b^2}{a-1} + \frac{a^2}{b-1}&=&\displaystyle \frac{\left(\frac{b^2}{a-1} + \frac{a^2}{b-1}\right)\cdot\left[(a-1)+(b-1)\right]}{a+b-2}\\\\
					&=&\displaystyle \frac{b^2+\frac{b^2(b-1)}{a-1}+\frac{a^2(a-1)}{b-1}+a^2}{a+b-2}\\\\
					&\geq&\displaystyle \frac{a^2+b^2+2\sqrt{\frac{b^2(b-1)}{a-1}\cdot \frac{a^2(a-1)}{b-1}}}{a+b-2}\\\\
					&=&\displaystyle \frac{a^2+b^2+2ab}{a+b-2}\\\\
					&=&\displaystyle \frac{(a+b)^2}{a+b-2}\\\\
					&=&\displaystyle \frac{[(a+b)^2]-4+4}{a+b-2}\\\\
					&=&\displaystyle \frac{(a+b+2)(a+b-2)+4}{a+b-2}\\\\
					&=&\displaystyle a+b+2+\frac{4}{a+b-2}\\\\
					&=&\displaystyle a+b-2+\frac{4}{a+b-2} + 4\\
					&\geq&\displaystyle 4+2\sqrt{4}\\
					&=&8,
				\end{array}
				$$

				等号成立当且仅当

				$$
				\left\{
				\begin{array}{rcl}
					\displaystyle \frac{b^2(b-1)}{a-1}&=&\displaystyle \frac{a^2(a-1)}{b-1}\\\\
					a+b-2&=&\displaystyle \frac{4}{a+b-2}\\
				\end{array}
				\right.
				$$

				即$a=b=1$.
			~\\

			\textbf{法四 (Cauchy–Schwarz 不等式)}: 

				$$
				\begin{array}{rcl}
					\displaystyle \frac{b^2}{a-1} + \frac{a^2}{b-1}&\geq&\displaystyle \frac{(a+b)^2}{a+b-2}\\\\
					&=&\displaystyle \frac{[(a+b)^2]-4+4}{a+b-2}\\\\
					&=&\displaystyle \frac{(a+b+2)(a+b-2)+4}{a+b-2}\\\\
					&=&\displaystyle a+b+2+\frac{4}{a+b-2}\\\\
					&=&\displaystyle a+b-2+\frac{4}{a+b-2} + 4\\
					&\geq&\displaystyle 4+2\sqrt{4}\\
					&=&8,
				\end{array}
				$$

				等号成立当且仅当

				$$
				\left\{
				\begin{array}{rcl}
					\displaystyle \frac{b}{a-1}&=&\displaystyle \frac{a}{b-1}\\\\
					a+b-2&=&\displaystyle \frac{4}{a+b-2}\\
				\end{array}
				\right.
				$$

				即$a=b=1$.

				\textbf{Cauchy–Schwarz 不等式的 Sedrakyan 表述}: $\displaystyle \sum_{i=1}^{n}{\frac{u^2}{v}}\geq\frac{\left(\sum_{i=1}^{n}{u}\right)^2}{\sum_{i=1}^{n}{v}}$, 等号成立当且仅当$\displaystyle \frac{u_1}{v_1}=\frac{u_2}{v_2}=\cdots=\frac{u_n}{v_n}.$

				\textbf{Cauchy–Schwarz 不等式}: $\displaystyle \sum_{i=1}^{n} u_i^2 \cdot \sum_{i=1}^{n} v_i^2 \geq \left(\sum_{i=1}^{n}u_i v_i\right)^2$, 等号成立当且仅当$\displaystyle \frac{u_1}{v_1}=\frac{u_2}{v_2}=\cdots=\frac{u_n}{v_n}.$

	\end{enumerate}

\end{document}