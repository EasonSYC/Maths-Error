%!TEX TX-program = xelatex
\documentclass[8pt]{article}

\usepackage{ctex}
\usepackage{graphicx}
\usepackage{enumitem}
\usepackage{geometry}
\usepackage{amsmath}
\usepackage{amssymb}
\usepackage{amsfonts}
\usepackage{tikz}
\usepackage{extarrows}
\usetikzlibrary{positioning}
\usetikzlibrary{svg.path}
\usepackage{xcolor}
\usepackage{soul}

\graphicspath{ {./images/} }

\author{高一(6)班\ 邵亦成\ 26号}
\title{8 综合练习(简单题)}
\date{2021年11月03日}

\geometry{a4paper, scale=0.8}

\begin{document}

	\maketitle

	\begin{enumerate}[label=(\arabic*)]
		\item 对于以下三个命题: (1) 若正实数$x, y$满足$y-xy=2$, 则$\displaystyle \frac{y}{x}$的最小值为8; (2) 若$x, y\in\mathbf{R}^{+}$且$x+2y-xy=0$, 则当且仅当$x=y=3$时, $xy$有最小值$9$; (3) 若$bc>ad, \displaystyle \frac{c}{a}>\frac{d}{b}$, 则$ab>0$, 其中所有真命题的序号为 (1) (3).
			~\\

			\textbf{考虑命题 (1)}: 

				$y-xy=2 \Rightarrow y=\displaystyle \frac{2}{1-x} \Rightarrow \frac{y}{x}=\frac{2}{x(1-x)}\geq\frac{2}{\frac{(x+1-x)^2}{4}}=8,$ 等号成立当且仅当$x=1-x$即$x=\displaystyle \frac{1}{2}$且$y=4$.

			\textbf{考虑命题 (2)}:

				$\displaystyle x+2y-xy=0 \Rightarrow y=\frac{x}{x-2} \Rightarrow x-2>0, xy=\frac{x^2}{x-2}.$

				令$t=x-2$则有$x=t+2$, $\displaystyle xy=\frac{t^2+4t+4}{t}=t+\frac{4}{t}+4\geq2\sqrt{t\cdot\frac{4}{t}}+4=8,$ 等号成立当且仅当$\displaystyle t=\frac{4}{t}$即$t=2$即$x=4$且$y=2$.

			\textbf{考虑命题 (3)}:

				假设$ab<0 \Rightarrow bc<ad$与已知条件矛盾, 于是$ab>0$.

		\item 设$a, b, c\in\mathbf{R}^{+},$
			\begin{enumerate}[label=(\arabic*)]
				\item 证明: $a^ab^b\geq\displaystyle (ab)^{\frac{a+b}{2}}.$
				\item 证明: $a^ab^bc^c\geq\displaystyle (abc)^{\frac{a+b+c}{3}}.$
			\end{enumerate}

			\textbf{法一: 变量在指数上, 考虑两边取对数}.

			\begin{enumerate}[label=(\arabic*)]
				\item 证明: $a^ab^b\geq\displaystyle (ab)^{\frac{a+b}{2}}.$

					即证$\ln{(a^a b^b)}\geq\displaystyle \ln{\left[(ab)^{\frac{a+b}{2}}\right]}$,

					即证$a\ln{a}+b\ln{b}\geq\displaystyle \frac{a+b}{2}\ln{(ab)}$,

					即证$a\ln{a}+b\ln{b}\geq\displaystyle \frac{a+b}{2}\ln{a}+\frac{a+b}{2}\ln{b}$,

					即证$\displaystyle \frac{a-b}{2} \ln{a} + \frac{b-a}{2} \ln{b} \geq 0$,

					即证$\displaystyle (a-b)(\ln a-\ln b)\geq 0$.

					由对称性, 不妨设$a\geq b$, 有$a-b \geq 0, \ln a-\ln b \geq 0$,

					于是$\displaystyle (a-b)(\ln a-\ln b)\geq 0$,

					即原不等式成立.

				\item 证明: $a^a+b^b+c^c\geq\displaystyle (abc)^{\frac{a+b+c}{3}}.$

					即证$\ln{(a^a b^b c^c)}\geq\displaystyle \ln{\left[(abc)^{\frac{a+b+c}{3}}\right]}$,

					即证$a\ln{a}+b\ln{b}+c\ln{c}\geq\displaystyle \frac{a+b+c}{3}\ln{(abc)}$,

					即证$a\ln{a}+b\ln{b}+c\ln{c}\geq\displaystyle \frac{a+b+c}{3}\ln{a}+\frac{a+b+c}{3}\ln{b}+\frac{a+b+c}{3}\ln{c}$,

					即证$\displaystyle (2a-b-c)\ln{a}+(2b-a-c)\ln{b}+(2c-a-b)\ln{c}\geq 0,$

					即证$(a-b)(\ln{a}-\ln{b})+(b-c)(\ln{b}-\ln{c})+(a-c)(\ln{a}-\ln{c})\geq 0.$

					由对称性, 不妨设$a\geq b \geq c$, 有$a-b \geq 0, \ln a-\ln b \geq 0, b-c \geq 0, \ln b-\ln c \geq 0, c-a \geq 0, \ln c-\ln a \geq 0$,

					于是$(a-b)(\ln{a}-\ln{b})+(b-c)(\ln{b}-\ln{c})+(a-c)(\ln{a}-\ln{c})\geq 0$,

					即原不等式成立.
			\end{enumerate}

			\textbf{法二: 两边均为正且带指数, 考虑作商.}


			\begin{enumerate}[label=(\arabic*)]
				\item 证明: $a^ab^b\geq\displaystyle (ab)^{\frac{a+b}{2}}.$

					两边作商, 有$\displaystyle \frac{a^a b^b}{(ab)^{\frac{a+b}{2}}}=a^{\frac{a-b}{2}}b^{\frac{b-a}{2}}=\left(\frac{a}{b}\right)^{\frac{a-b}{2}}.$

					由对称性, 不妨设$a\geq b$, 有$\displaystyle \frac{a}{b}>1, \frac{a-b}{2}>0$, 由幂的基本不等式有$\displaystyle \left(\frac{a}{b}\right)^{\frac{a-b}{2}}\geq 1.$

					于是有$a^ab^b\geq\displaystyle (ab)^{\frac{a+b}{2}}.$

				\item 证明: $a^a+b^b+c^c\geq\displaystyle (abc)^{\frac{a+b+c}{3}}.$

					两边作商, 有$\displaystyle \frac{a^a b^b c^c}{(ab)^{\frac{a+b+c}{3}}}=a^{\frac{2a-b-c}{3}}b^{\frac{2b-a-c}{3}}c^{\frac{2c-a-b}{3}}=\left(\frac{a}{b}\right)^{\frac{a-b}{3}} \left(\frac{b}{c}\right)^{\frac{b-c}{3}} \left(\frac{a}{c}\right)^{\frac{a-c}{3}}.$

					由对称性, 不妨设$a\geq b\geq c$, 有$\displaystyle \frac{a}{b}>1, \frac{a-b}{3}>0, \frac{b}{c}>1, \frac{b-c}{3}>0, \frac{a}{c}>1, \frac{a-c}{3}>0$, 由幂的基本不等式有$\displaystyle \left(\frac{a}{b}\right)^{\frac{a-b}{2}}\geq 1, \left(\frac{b}{c}\right)^{\frac{b-c}{2}}\geq 1, \left(\frac{a}{c}\right)^{\frac{a-c}{2}}\geq 1.$

					于是有$a^a+b^b+c^c\geq\displaystyle (abc)^{\frac{a+b+c}{3}}.$

			\end{enumerate}
	\end{enumerate}

\end{document}