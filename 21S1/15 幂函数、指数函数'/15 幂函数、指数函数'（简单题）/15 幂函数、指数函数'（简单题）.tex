%!TEX TX-program = xelatex
\documentclass[8pt]{article}

\usepackage{ctex}
\usepackage{graphicx}
\usepackage{enumitem}
\usepackage{geometry}
\usepackage{amsmath}
\usepackage{amssymb}
\usepackage{amsfonts}
\usepackage{tikz}
\usetikzlibrary{positioning}
\usetikzlibrary{svg.path}
\usetikzlibrary{fit}
\usepackage{xcolor}
\usepackage{longtable}

\graphicspath{ {./images/} }

\newcommand\addvmargin[1]{
  \node[fit=(current bounding box),inner ysep=#1,inner xsep=0]{};
}

\setlength{\tabcolsep}{5mm} % separator between columns
\def\arraystretch{1.25} % vertical stretch factor

\author{高一(6)班\ 邵亦成\ 26号}
\title{15 幂函数、指数函数'}
\date{2021年12月29日}

\geometry{a4paper, scale=0.8}

\begin{document}

	\maketitle

	\begin{enumerate}[label=\arabic*.]
		\item 已知函数$f(x)=\left\vert 2^x - 1\right\vert - \left\vert 2^x + 1 \right\vert - a - 1$恒有零点, 则$a$的取值范围为?.
			~\\

			即方程

			$$\left\vert 2^x - 1\right\vert - \left\vert 2^x + 1 \right\vert = a + 1$$

			有解.

			令$g(x)=\left\vert 2^x - 1\right\vert - \left\vert 2^x + 1 \right\vert,$化简$g(x)$, 有

			$$g(x)=\left\vert 2^x - 1\right\vert - 2^x - 1 =\left\{\begin{array}{rl}-2\cdot 2^x, &x<0,\\-2, &x\geq 0.\end{array}\right.$$

			绘制$y=g(x)$的图像,

			$$
			\begin{tikzpicture}[scale=1.5, baseline=0]
	    		\draw[black, ->] (-5,  0)--( 5,  0) node[below] {$x$};
	    		\draw[black, ->] ( 0, -3)--( 0,  1) node[right] {$y$};
	    		\draw[black, domain=0:4, samples=100] plot(\x, -2) node[right] {$y=g(x)$};
	    		\draw[black, domain=-4:0, samples=100] plot(\x, {-2 * 2^(\x)});
	    		\filldraw[black] (0, -2) circle (0.1em) node[anchor=south west] {$(0, -2)$};
	    		\filldraw[black] (0, 0) circle (0.1em) node[anchor=south west] {$O$};
				\addvmargin{1mm}
			\end{tikzpicture}
			$$

			显然有

			$$\lim_{x \rightarrow -\infty} g(x) = 0, -2 \leq g(x) < 0.$$

			故

			$$a+1\in[-2, 0),$$

			故

			$$a\in[-3, -1).$$

		~\\

		\item 已知定义域为$[0, 1]$的函数$f(x)$同时满足: (1) $\forall x\in[0, 1], f(x)\geq 0$. (2) $f(1) = 1$. (3) $\forall x_1 \geq 0, x_2 \geq 0, x_1 + x_2 \leq 1, f(x_1 + x_2) \geq f(x_1) + f(x_2)$.
			\begin{enumerate}[label=(\arabic*)]
				\item 求$f(0)$的值.
					~\\

					令$x_1 = 1, x_2 = 0$, 有

					$$f(1) \geq f(1) + f(0), f(0) \leq 0.$$

					又

					$$\forall x\in[0, 1], f(x) \geq 0,$$

					有

					$$f(0)=0$$

				~\\

				\item 求$f(x)$的最大值.
					~\\

					下证: $f(x)$在$[0, 1]$上为增函数.

					$\forall 0\leq x_1 < x_2 \leq 1$,

					$$f(x_2) - f(x_1) \geq f(x_2 - x_1) \geq 0, f(x_2) \geq f(x_1),$$

					故有

					$$\max_{x\in[0, 1]} f(x) = f(1) = 1.$$

				~\\

				\item 若$\forall x\in[0, 1)$, $4f^2(x) - 4(2-a) f(x) + 5-4a \geq 0$, 求$a$的取值范围.
					~\\

					$x\in [0, 1), f(x) \in [0, 1].$

					令$g(x) = 4x^2 - 4(2-a)x + 5-4a,$ $y=g(x)$对称轴为直线$x=\dfrac{2-a}{2}$.

					原命题等价于

					$$\min_{x\in[0, 1]} g(x) \geq 0.$$

					分三类讨论.

					\begin{enumerate}[label=$\arabic*^{\circ}$]
						\item $\dfrac{2-a}{2} \leq 0$, $a \geq 2$, 

							$$\min_{x\in[0, 1]} g(x) = g(0) = 5-4a \geq 0,$$

							$a\leq \dfrac{5}{4}$, 故$a\in \emptyset$.

						\item $0 < \dfrac{2-a}{2} \leq 1$, $0 \leq a < 2$,

							$$\min_{x\in[0, 1]} g(x) = g\left(\frac{2-a}{2}\right) \geq 0,$$

							$a\in[-1, 1]$, 故$a\in [0, 1]$.

						\item $1 < \dfrac{2-a}{2}$, $a<0$,

							$$\min_{x\in[0, 1]} g(x) = g(1) = 1 \geq 0,$$

							$a \in \mathbb{R}$, 故$a \in (-\infty, 0)$.

					\end{enumerate}

					综上所述, $a\in (-\infty, 1]$.

			\end{enumerate}

	\end{enumerate}

\end{document}