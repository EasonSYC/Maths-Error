%!TEX TX-program = xelatex
\documentclass[8pt]{article}

\usepackage{ctex}
\usepackage{graphicx}
\usepackage{enumitem}
\usepackage{geometry}
\usepackage{amsmath}
\usepackage{amssymb}
\usepackage{amsfonts}
\usepackage{tikz}
\usetikzlibrary{positioning}
\usetikzlibrary{svg.path}
\usetikzlibrary{fit}
\usepackage{xcolor}
\usepackage{longtable}

\graphicspath{ {./images/} }

\newcommand\addvmargin[1]{
  \node[fit=(current bounding box),inner ysep=#1,inner xsep=0]{};
}

\setlength{\tabcolsep}{5mm} % separator between columns
\def\arraystretch{1.25} % vertical stretch factor

\author{高一(6)班\ 邵亦成\ 26号}
\title{114 幂函数、指数函数(简单题)}
\date{2021年12月22日}

\geometry{a4paper, scale=0.8}

\begin{document}

	\maketitle

	\begin{enumerate}[label=\arabic*.]
		\item 对于幂函数$f(x)=x^{\frac{4}{5}}$, 若$0<x_1<x_2$, 则$\displaystyle f\left(\frac{x_1 + x_2}{2}\right), \frac{f(x_1)+f(x_2)}{2}$大小关系是?.
			~\\

			绘制$f(x)=x^{\frac{4}{5}}$的图像如下图:

			$$\begin{tikzpicture}[scale=1.5, baseline=0]
	    		\draw[black, ->] (-1,  0)--( 6,  0);
	    		\draw[black, ->] ( 0, -1)--( 0,  5);
	    		\draw[black, domain=0:5] plot(\x, {\x^(4/5)});
	    		\filldraw[black] (1, 1) circle (0.1em) node[anchor=north west] {$x_1$};
	    		\filldraw[black] (3, {3^(4/5)}) circle (0.1em) node[anchor=north west] {$x_2$};
	    		\draw[black] (1,1)--(3, {3^(4/5)});
	    		\draw[black, dashed] (2, {2^(4/5)})--(2, 0) node[below] {$\dfrac{x_1+x_2}{2}$};
	    		\addvmargin{1mm}
	    	\end{tikzpicture}$$

	    	$\displaystyle f\left(\frac{x_1 + x_2}{2}\right)$即为曲线与虚线的交点, $\displaystyle \frac{f(x_1)+f(x_2)}{2}$即为直线与虚线的交点. $f(x)=x^{\frac{4}{5}}$在$[x_1, x_2]$上为上凸函数 (即在$[x_1, x_2]$上$f''(x)<0$), 故$\displaystyle f\left(\frac{x_1 + x_2}{2}\right) > \frac{f(x_1)+f(x_2)}{2}$.

	\end{enumerate}

\end{document}