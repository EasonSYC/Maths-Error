%!TEX TX-program = xelatex
\documentclass[8pt]{article}

\usepackage{ctex}
\usepackage{graphicx}
\usepackage{enumitem}
\usepackage{geometry}
\usepackage{amsmath}
\usepackage{amssymb}
\usepackage{amsfonts}
\usepackage{tikz}
\usetikzlibrary{positioning}
\usetikzlibrary{svg.path}
\usetikzlibrary{fit}
\usepackage{xcolor}
\usepackage{longtable}

\graphicspath{ {./images/} }

\newcommand\addvmargin[1]{
  \node[fit=(current bounding box),inner ysep=#1,inner xsep=0]{};
}

\setlength{\tabcolsep}{5mm} % separator between columns
\def\arraystretch{1.25} % vertical stretch factor

\author{高一(6)班\ 邵亦成\ 26号}
\title{14 幂函数、指数函数(难题)}
\date{2021年12月22日}

\geometry{a4paper, scale=0.8}

\begin{document}

	\maketitle

	\begin{enumerate}[label=\arabic*.]
		\item 已知函数$f(x)$是$(-\infty, +\infty)$上的奇函数, 且$f(x)$的图像关于直线$x=1$对称, 当$x\in [-1, 0]$时, $f(x)=-x$, 则$f(1)+f(2)+f(3)+\cdots+f(2015)=$?.
			~\\

			$f(x)$的图像关于直线$x=1$对称, 即

			$$f(1-x)=f(1+x),$$

			又$f(x)$是奇函数, 有

			$$-f(x-1)=f(x+1),$$

			即

			$$f(x)=-f(x+2),$$

			即

			$$f(x)=f(x+4).$$

			绘制$y=f(x)$在$[-1, 3]$上的图像如图所示:

			$$\begin{tikzpicture}[scale=1.5, baseline=0]
	    		\draw[black, ->] (-2,  0)--( 4,  0);
	    		\draw[black, ->] ( 0, -2)--( 0,  2);
	    		\draw[black] (-1, 1)--(1, -1)--(3, 1);
	    		\node[anchor=north east] at (0, 0) {$O$};
	    		\draw[black, dashed] (1, -1.5)--(1, 1.5);
	    		\node[anchor=north east] at (1, 0) {$1$};
	    		\draw[black, dashed] (-1, 1)--(-1, 0);
	    		\node[anchor=north] at (-1, 0) {$-1$};
	    		\draw[black, dashed] (3, 1)--(3, 0);
	    		\node[anchor=north] at (3, 0) {$3$};
	    		\addvmargin{1mm}
	    	\end{tikzpicture}$$

			故有$f(1)=-f(-1)=-1, f(2)=f(0)=0, f(3)=1, f(4)=f(0)=0,$

			$$f(1)+f(2)+f(3)+\cdots+f(2015)=503\times[f(1)+f(2)+f(3)+f(4)]+f(1)+f(2)+f(3)=0.$$

		~\\

		\item 已知$f(x)=\dfrac{2x-m}{x^2+1}$是定义在实数集$\mathbb{R}$上的函数, 把方程$f(x)=\dfrac{1}{x}$称为函数$f(x)$的特征方程, 特征方程的两个实根$\alpha, \beta (\alpha < \beta)$称为$f(x)$的特征根.
			\begin{enumerate}[label=(\arabic*)]
				\item 讨论函数的奇偶性, 并说明理由.
					~\\

					略, $m=0$为奇函数, $m\neq 0$为非奇非偶函数.

				~\\

				\item 求$\alpha f(\beta) + \beta f(\alpha)$的值.
					~\\

					方程

					$$f(x)=\frac{1}{x}$$

					的根即为方程

					$$x^2 - mx - 1 = 0$$

					的非零根, 即为该方程的根.
					~\\

					于是有

					$$\Delta = m^2 + 4 > 0.$$

					由韦达定理, 有

					$$\alpha + \beta = m, \alpha \beta = -1.$$

					故

					\begin{align*}
						\text{原式} &= \alpha f(\beta) + \beta f(\alpha)\\
						&= \frac{\alpha}{\beta} + \frac{\beta}{\alpha}\\
						&= \frac{(\alpha + \beta)^2 - 2\alpha \beta}{\alpha \beta}\\
						&= -m^2 - 2.
					\end{align*}

				~\\

				\item 判断函数$y=f(x), x\in[\alpha, \beta]$的单调性, 并证明.
					~\\

					考虑$\alpha \leq x_1 < x_2 \leq \beta$,

					\begin{align*}
						f(x_1) - f(x_2) &= \frac{2x_1 - m}{x_1^2 + 1} - \frac{2x_2 - m}{x_2^2 + 1}\\
						&= \frac{(x_2 - x_1)[2x_1 x_2 - m(x_1 + x_2) - 2]}{(x_1^2 + 1)(x_2^2 + 1)}.
					\end{align*}

					$$\alpha \leq x_1 < x_2 \leq \beta \Rightarrow x_2 - x_1 > 0, x_1^2 + 1 > 0, x_2^2 + 1 > 0, x_1^2 - mx_1 - 1 < 0, x_2^2 - mx_2 - 1 < 0, 2x_1 x_2 - m(x_1 + x_2) < 0,$$

					故

					$$f(x_1) - f(x_2) = \frac{(x_2 - x_1)[2x_1 x_2 - m(x_1 + x_2) - 2]}{(x_1^2 + 1)(x_2^2 + 1)} < 0,$$

					即$f(x)$在$[\alpha, \beta]$内严格增.

			\end{enumerate}

	\end{enumerate}

\end{document}