%!TEX TX-program = xelatex
\documentclass[8pt]{article}

\usepackage{ctex}
\usepackage{graphicx}
\usepackage{enumitem}
\usepackage{geometry}
\usepackage{amsmath}
\usepackage{amssymb}
\usepackage{amsfonts}
\usepackage{tikz}
\usepackage{extarrows}
\usetikzlibrary{positioning}
\usetikzlibrary{svg.path}
\usepackage{xcolor}
\usepackage{soul}

\graphicspath{ {./images/} }

\author{高一(6)班\ 邵亦成\ 26号}
\title{11 函数的基本性质(简单题)}
\date{2021年12月1日}

\geometry{a4paper, scale=0.8}

\begin{document}

	\maketitle

	\begin{enumerate}[label=\arabic*.]
		\item $f(x)$是偶函数, $g(x)$是奇函数, 它们的定义域都是$\{x|x\neq \pm 1, x\in \mathbb{R}\}$且满足$f(x)+g(x)=\dfrac{1}{x-1}$, 则$f(x)=$?, $g(x)=$?.
			~\\

			答案为$\dfrac{1}{x^2-1}; \dfrac{x}{x^2-1}$并不难求出, 但本题的价值在可以根据此题写出一个定理.

			\textbf{定理  } 对于一个定义在$D$上的函数$h(x)$, 如果有$\forall x\in D: -x \in D$, 则$h(x)$能被写为同样定义在$D$上的两个函数$f(x)$和$g(x)$的和, 且$f(x)$为偶函数, $g(x)$为奇函数. 其中, $$f(x)=\dfrac{h(x)+h(-x)}{2}, g(x)=\dfrac{h(x)-h(-x)}{2}.$$

		~\\

		\item 若函数$f(x)=a|x-b|+2$在$x\in[0, +\infty)$上为减函数, 则实数$a, b$的取值范围是?.
			~\\

			\begin{enumerate}[label=$\arabic*^{\circ}$]
				\item $a=0$, 有$f(x)=2$, 符合.

				\item $a\neq 0$, 有

				$$f(x)=\left\{\begin{array}{rl}ax-ab+2, &x\geq b,\\-ax+ab-2, &x<b.\\\end{array}\right.$$

				故有$a>0, b\leq 0$.
			\end{enumerate}

			综上, $(a, b)\in \{(x, y)|x=0, y\in\mathbb{R} \text{ or } x>0, y\leq 0\}$.

		~\\

		\item 设$T=x^2-(m+1)x-m$,
			\begin{enumerate}[label=(\arabic*)]
				\item 当$x\in[0, 2]$时, 恒有$T>0$, 求$m$的取值范围.
					~\\

					记

					$$f(x)=T=x^2-(m+1)x-m, \Delta = [-(m+1)]^2-4(-m)=m^2+6m+1$$

					\textbf{法一}: 二次函数的判别式法

					有

					$$\Delta < 0 \text{ or } \left\{\begin{array}{rcl}\Delta&=&0\\\dfrac{m+1}{2}<0&\text{ or }&\dfrac{m+1}{2}>2\end{array}\right. \text{ or } \left\{\begin{array}{rcl}\Delta&>&0\\\dfrac{m+1}{2}<0&\text{ or }&\dfrac{m+1}{2}>2\\f(0)&>&0\\f(2)&>&0\end{array}\right. $$

					即

					$$m\in\left(-\infty, -3+2\sqrt{2}\right).$$

					\textbf{法二}: 二次函数的极值问题

					有

					$$\min_{x\in[0, 2]} f(x) > 0,$$

					即

					$$\left\{\begin{array}{rcl}\dfrac{m+1}{2}&<&0\\f(0)&>&0\end{array}\right. \text{ or } \left\{\begin{array}{rcl}\dfrac{m+1}{2}&>&2\\f(2)&>&0\end{array}\right. \text{ or } \left\{\begin{array}{rcl}\dfrac{m+1}{2}&\in&[0,2]\\f\left(\dfrac{m+1}{2}\right)&>&0\end{array}\right. $$

					即

					$$m\in\left(-\infty, -3+2\sqrt{2}\right).$$

					\textbf{法三}: 参变分离

					有

					$$m<\frac{x^2-x}{x+1},$$

					即

					$$m<\min_{x\in[0, 2]} \frac{x^2-x}{x+1},$$

					而

					\begin{align*}
						\frac{x^2-x}{x+1} &= x - 2 + \frac{2}{x+1}\\
						&= (x+1) + \frac{2}{x+1} - 3\\
						&\geq -3 + 2\sqrt{2},
					\end{align*}

					等号成立当且仅当$x+1=\sqrt{2}$即$x=\sqrt{2}-1\in[0, 2]$,

					故

					$$m\in\left(-\infty, -3+2\sqrt{2}\right).$$

				~\\

				\item 当$m\in[0, 2]$时, 恒有$T>0$, 求$x$的取值范围.
					~\\

					记

					$$g(m)=T=x^2-(m+1)x-m=m(-x-1)+x^2-x$$

					\textbf{一次函数的图像}:

					有

					$$\left\{\begin{array}{rcl}g(0)&>&0\\g(2)&>&0\end{array}\right.$$

					即

					$$x\in\left(-\infty, \frac{3-\sqrt{17}}{2}\right) \cup \left(\frac{3+\sqrt{17}}{2}, +\infty\right).$$

			\end{enumerate}

	\end{enumerate}

\end{document}