%!TEX TX-program = xelatex
\documentclass[8pt]{article}

\usepackage{ctex}
\usepackage{graphicx}
\usepackage{enumitem}
\usepackage{geometry}
\usepackage{amsmath}
\usepackage{amssymb}
\usepackage{amsfonts}
\usepackage{tikz}
\usepackage{extarrows}
\usetikzlibrary{positioning}
\usetikzlibrary{svg.path}
\usepackage{xcolor}
\usepackage{soul}

\graphicspath{ {./images/} }

\author{高一(6)班\ 邵亦成\ 26号}
\title{11 函数的基本性质(难题)}
\date{2021年12月1日}

\geometry{a4paper, scale=0.8}

\begin{document}

	\maketitle

	\begin{enumerate}[label=\arabic*.]
		\item 已知函数$f(x)$是$\mathbb{R}$上的严格增函数, $A(0, -1), B(3, 1)$是其图像上的两点, 那么$|f(x+1)|<1$的解集是?.
			~\\

			$$
			\begin{tikzpicture}[scale=1.5, baseline=0]
	    		\draw[black, ->] (-2,  0)--( 4,  0);
	    		\draw[black, ->] ( 0, -2)--( 0,  2);
	    		\draw[black] (0, -1)--(3, 1) node[right] {$y=f(x)$};
	    		\draw[blue] (2, 1)--(-1, -1) node[left] {$y=f(x+1)$};
	    		\draw[red] (2, 1)--(0.5, 0)--(-1, 1) node[left] {$y=|f(x+1)|$};
	    		\filldraw[black] (0, -1) circle (0.1em) node[anchor=east] {$-1$} node[anchor=west] {$A$};
	    		\filldraw[black] (3, 1) circle (0.1em) node[anchor=south] {$(3,1)$} node[anchor=north] {$C$};
	    		\filldraw[red] (2, 1) circle (0.1em) node[anchor=south] {$(2,1)$};
	    		\filldraw[blue] (-1, -1) circle (0.1em) node[anchor=north] {$(-1, -1)$};
	    		\filldraw[red] (-1, 1) circle (0.1em) node[anchor=south] {$(-1, 1)$};
	    	\end{tikzpicture}$$

	    	故答案为$(-1, 2).$

    	~\\

    	\item 设$f(x)$是定义在$\mathbb{R}$上的偶函数, 且图像关于$x=2$对称. 已知$x\in [0, 2]$时, $f(x)=-\sqrt{x}+1$.
    		\begin{enumerate}[label=(\arabic*)]
    			\item 证明$f(x)$是周期函数, 并求出它的一个周期.
    				~\\

					有

					$$f(2+x)=f(2-x)=f(x-2)$$

					即

					$$f(4+x)=f(x)$$

					即$f(x)$是周期函数, 且存在一个周期$T=4$.

				~\\

				\item 求$x\in [-6, -2]$时, $f(x)$的表达式.
					~\\

					考虑$x\in[-2, 0)$时$f(x)$的解析式.

					当$x\in [-2, 0)$时, 有$-x \in (0, 2]$, 且

					$$f(x)=f(-x)=-\sqrt{-x}+1.$$

					故有

					$$f(x)=\left\{\begin{array}{rcl}-\sqrt{x}+1, &x\in[0, 2]\\-\sqrt{-x}+1, &x\in[-2, 0).\end{array}\right.$$

					考虑$x\in[-6, -2]$时$f(x)$的解析式.

					当$x\in [-6, -2]$时, 有$x+4 \in [-2, 2]$, 且

					$$f(x)=f(x+4)=\left\{\begin{array}{rcl}-\sqrt{x+4}+1, &x\in[-4, -2]\\-\sqrt{-x-4}+1, &x\in[-6, -4).\end{array}\right.$$

					即

					$$f(x)=\left\{\begin{array}{rcl}-\sqrt{x+4}+1, &x\in[-4, -2]\\-\sqrt{-x-4}+1, &x\in[-6, -4).\end{array}\right.$$

	    	\end{enumerate}

	\end{enumerate}

\end{document}